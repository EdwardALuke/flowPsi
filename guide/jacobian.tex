\subsection{Jacobian Formulations}

The Jacobian matrix used in the Newton method, described by
Eq.~(\ref{Jacobian}), consists of a block-diagonal matrix, formed from
Jacobians of chemistry source terms, combined with both diagonal and
off-diagonal matrices, formed from the differentiation of flux functions.
When constructing this Jacobian, one takes advantage of the fact that
the flux function, $\hat{F}(q_{l}, q_{r})$,
is defined by the conservative variables on the left and
right side of the face.  Then the Jacobian of this function will
consist of two matrices, given by
\begin{equation}
\begin{split}
fjl & = \mathcal{A}^{n+1}\frac{\partial \hat{F}(q_{l},q_{r})}{\partial q_{l}},\\
fjr & = \mathcal{A}^{n+1}\frac{\partial \hat{F}(q_{l},q_{r})}{\partial q_{r}}.
\end{split}
\end{equation}
The variables above, {\tt fjl} and {\tt fjr}, are Jacobians of the flux
functions located at faces and are components of the overall Jacobian
matrix given in Eq.~(\ref{Jacobian}).
More details on the final form of the Jacobian matrix are given by 
Luke\cite{luke.99}.

\subsubsection{Inviscid Flux Jacobians}

In the study, all Jacobians are evaluated analytically, except for the
inviscid flux, whereby Jacobians associated with the Roe scheme are
too complex to implement efficiently.  In this case, several
alternatives are available: 1) use analytic Jacobians of simpler
inviscid flux functions such as Steger-Warming or Van Leer, or 2)
compute Jacobians of the Roe flux numerically. The
analytic Van Leer Jacobian was found to provide superior stability properties,
but severely hampered convergence of the implicit scheme.  On the
other hand, the numerical Roe Jacobians provided better convergence
rates, but sometimes produced ill-conditioned linear systems.  It was
observed that these problems occurred in regions where discontinuities
existed in the Roe Jacobians, due to the use of non-linear absolute
value functions.  A compromise that appears to provide a reasonable
balance between convergence and robustness has been found by smoothing
the Roe flux function before taking its derivative.  This is
accomplished by replacing the absolute value function $|x|$ with the continuous
function $\sqrt{x^2+\epsilon}$, where $\epsilon$ is set to $.05$ times
the average sound speed.

\subsubsection{Viscous Flux Jacobians}

In the construction of viscous flux Jacobians, a thin-layer
approximation is used in the representation of the face gradient by
using simple centered differences in the direction of the vector
connecting cell centroids on either side of the face.

\subsubsection{Jacobians for Turbulence Models}

In order to preserve the diagonal dominance of the matrix in the
linear system, positive parts of the source terms are not linearized:
only the negative parts (which model the destruction of turbulence)
are included in the Jacobians.  For the Jacobians used in the BSL,
SST, and $k-\omega$ models the approach of Merci {\sl et
  al.\/}\cite{Merci.00} is adopted.

